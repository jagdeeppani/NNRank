\documentclass{article}

% use Times
\usepackage{times}
% For figures
%\usepackage{epsfig} % less modern
%\usepackage{subfig}

% For citations
\usepackage{natbib}

% For algorithms
\usepackage{algorithm}
\usepackage{algorithmic}

% As of 2011, we use the hyperref package to produce hyperlinks in the
% resulting PDF.  If this breaks your system, please commend out the
% following usepackage line and replace \usepackage{icml2016} with
% \usepackage[nohyperref]{icml2016} above.
\usepackage{hyperref}

% Packages hyperref and algorithmic misbehave sometimes.  We can fix
% this with the following command.
\newcommand{\theHalgorithm}{\arabic{algorithm}}

% Employ the following version of the ``usepackage'' statement for
% submitting the draft version of the paper for review.  This will set
% the note in the first column to ``Under review.  Do not distribute.''
\usepackage{icml2016}
%
%
%\usepackage{subfigure}
\usepackage{epstopdf}
\usepackage{wrapfig}

\usepackage{caption}
% \usepackage[skip=0cm,list=true,labelfont=it]{subcaption}
\usepackage{framed}
\usepackage{color}
\usepackage{amssymb}
\usepackage{amsmath}
\usepackage{amsthm}
\usepackage{graphicx}
\usepackage{multirow}
% \usepackage{mathtools}
\usepackage{url}
\usepackage{booktabs}
\usepackage{verbatim}
% \usepackage[numbers,sort&compress]{natbib}
\usepackage{booktabs}
\usepackage{wrapfig}
\usepackage{epstopdf}
\usepackage{subcaption}
\usepackage{cleveref}
\usepackage{verbatim}
\usepackage{xcolor}
\usepackage{relsize}
% Employ this version of the ``usepackage'' statement after the paper has
% been accepted, when creating the final version.  This will set the
% note in the first column to ``Proceedings of the...''
%\usepackage[accepted]{icml2016}

\newcommand{\plmi}{$\pm\;$}


\newtheorem{conj}{Conjecture}
\newtheorem{lemma}{Lemma}
\newtheorem{cor}{Corollary}
\newtheorem{theorem}{Theorem}
\newtheorem{corollary}{Corollary}
\newtheorem{definition}{Definition}
\newtheorem{example}{Example}
\newtheorem{question}{Question}
\newtheorem{assumption}{Assumption}
\newtheorem{remark}{Remark}






\newcommand{\nn}{\nonumber}
\newcommand{\mc}{\mathcal}
\newcommand{\mb}{\mathbf}
\newcommand{\mbb}{\mathbb}
\newcommand{\mr}{\mathring}
\newcommand{\bd}{\bar{\delta}}
\newcommand{\tc}{\textsc}
\newcommand{\ceqref}{Eq.~\eqref}
\newcommand{\ceqrefs}{Eqs.~\eqref}
\newcommand{\mcite}{\citep}
\newcommand{\mmcite}{\citet}


\newcommand{\para}{\textbf}
\newcommand{\parae}{\emph}


\renewcommand{\algorithmicrequire}{\textbf{Input:}}
\renewcommand{\algorithmicensure}{\textbf{Output:}}

\def\name{{\bf TSVDNMF~}}
\def\noise{~Heavy noise~}
\def\prob{\hbox{Pr}}
\def\Min{\text{Min}}
\def\Max{\text{Max}}
\def\bA{{\bf A}}
\def\bB{{\bf B}}
\def\bC{{\bf C}}
\def\bM{{\bf M}}
\def\bW{{\bf W}}
\def\bP{{\bf P}}
\def\RR{\mathbb R}

\def\mg{{\mathcal G}}
\def\bm{{\bf m}}
\def\br{{\bf r}}

\def\EE{{\mathbb{E}}}
\def\I{{\mathbb{I}}}
\def\bX{{\bf X}}
\def\bx{{\bf x}}
\def\bz{{\bf z}}
\def\bv{{\bf v}}
\def\bb{{\bf b}}
\def\bt{{\bf t}}
\def\bk{{\bf k}}
\def\bn{\bar{n}}
\def\bN{\bf{N}}
\def\bd{\bar{d}}
\def\g{\tc{g}}
\def\G{\tc{g}}
\def\Q{\tc{q}}
\def\B{\tc{b}}
\def\P{\tc{p}}
\def\bP{{\bf P}}
\def\H{\tc{h}}
\def\X{\tc{x}}
\def\A{\tc{a}}
\def\L{\tc{l}}
\def\Y{\tc{y}}
\def\C{\tc{c}}
\def\M{\tc{m}}
\def\N{\tc{n}}
\def\J{\tc{j}}
\def\R{\tc{r}}
\def\E{\tc{e}}
\def\K{\tc{k}}
\def\l2svd{{\ell^2_2\text{TSVD}}}
%\DeclareMathOperator{\conv}{conv}
%\DeclareMathOperator{\argmax}{argmax}
%\DeclareMathOperator{\argmin}{argmin}
%\DeclareMathOperator{\cone}{cone}
\DeclareMathOperator{\rank}{rank}
%\DeclareMathOperator{\var}{Var}
%\DeclareMathOperator{\cov}{Cov}
%\DeclareMathOperator{\col}{col}
%\DeclareMathOperator{\diag}{diag}
%\DeclareMathOperator{\tr}{tr}

\providecommand{\norm}[1]{\lVert#1\rVert}

\def\BAD{\hbox{BAD}}
\def\for{\hbox{ for }}
\def\var{\mbox{Var}}

\newtheorem{claim}{Claim}[section]

\newcommand{\details}[1]{{\color{blue}\ #1 }} %%This is the text that needs to be revised
\newcommand{\nnote}[1]{{\color{red}\ [Navin: #1 ]}}

% The \icmltitle you define below is probably too long as a header.
% Therefore, a short form for the running title is supplied here:
\icmltitlerunning{Computing Non-negative Rank}

\begin{document}

\twocolumn[
\icmltitle{Computing Non-negative Rank}

% It is OKAY to include author information, even for blind
% submissions: the style file will automatically remove it for you
% unless you've provided the [accepted] option to the icml2016
% package.
\icmlauthor{Your Name}{email@yourdomain.edu}
\icmladdress{Your Fantastic Institute,
            314159 Pi St., Palo Alto, CA 94306 USA}
\icmlauthor{Your CoAuthor's Name}{email@coauthordomain.edu}
\icmladdress{Their Fantastic Institute,
            27182 Exp St., Toronto, ON M6H 2T1 CANADA}

% You may provide any keywords that you
% find helpful for describing your paper; these are used to populate
% the "keywords" metadata in the PDF but will not be shown in the document
\icmlkeywords{boring formatting information, machine learning, ICML}
\vskip 0.3in
]

\begin{abstract}

\end{abstract}









\section{Introduction}







\section{Experiments}

\subsection{Synthetic experiments}
In this section our goal is to show that the proposed algorithm discovers the correct non-negative rank (up to approximation) when the underlying dominant assumptions hold on the dataset. We present results under two noise models, Gaussian and Multinomial noise models. The dataset is assumed to be generated under dominant assumption.

\textit{Dominant data:} 
%Data is generated under the dominant basis vectors and dominant features assumptions.
Each column of $C$ is generated from a symmetric Dirichlet distribution with hyper-parameter $\frac{1}{2k}$.
%$\footnote{nearly $77\%$ of the data points satisfy dominant basis vector assumption ($\alpha>0.4, \beta<0.3$)}.
Columns of $B$ are also generated from Dirichlet by randomly selecting $c$ features and putting weight $\propto \eta_0$ on these features.
Concretely, let $c$ be the number of dominant features for each basis vector and $\eta_{0}$ be the sum of the weights of the dominant features in each basis vector.
Assume $\eta = (\frac{\eta_{0}}{1-\eta_{0}})\cdot(\frac{d-c}{c})$.
For the $j$th basis vector, we set $\widetilde{\alpha}_{j} = \textbf{1}_{d \times 1}$ and multiply the elements of $\widetilde{\alpha}_{j}$ indexed from $(c(j-1)+1) \hspace{1mm}\text{to}\hspace{1mm} cj$ by $\eta$.
% Vector $B_{j}$, 
$j$th column of $B$, is then generated from a Dirichlet with hyper-parameter $\widetilde{\alpha}_{j}$. We chose $c = 3, \eta_0=0.1.$

\textit{Gaussian Noise: } 
Noise matrix $N$ is generated with each entry from $\mathcal{N}(0,\sigma_{i})$ where $\sigma_{i}=\beta \times B_{il}$. $B_{il}$ is the maximum element in $i$th row of $B$.
This is based on the subset noise. We chose $\beta \in [0 , 0.5]$ as the noise level.

\textit{Multinomial Noise: }
%$N_{.j}$, $j$th column of noise matrix $N$, is generated by sampling an $i \in [1\cdots d]$ with prob (picking $i$) $= (BC)_{ij}$ $m$ times and taking the average to find $\tilde{N}_j$. Then $N_j$ is set as $\tilde{N}_j-(BC)_j$ to set the mean of noise to zero. 
The matrices $B$ and $C$ are normalized such that each of their column sums to one.
$N_{.j}$, $j$th column of noise matrix $N$, is generated by picking $m$ samples (a sample is a 1-of-d coding) from $ [1\cdots d]$ with prob($i$) $= (BC)_{ij}$ and taking an average of the samples to find $\tilde{N}_{.j}$. Then $N_{.j}$ is set as $\tilde{N}_{.j}-(BC)_{.j}$.
%to set the mean of noise $N_{.j}$ to zero. 
Lower $m$ implies high noise and vice versa, $m \in \{10,60,100\}$.
%{\bf Benchmarks:} We compare  


\textit{Parameters: } $edgeThreshold = 3 \times \frac{n}{s}$.  $\gamma = 0.5$. For higher noise level, the number of groups $s$ varies as per the table.
In the experiments we consider $d=1000, n=2000, k=20$.
Table ~\ref{table:Gaussian} presents the quality of non-negative rank discovered under Gaussian noise assumption. The rank presented is the median over 20 different datasets generated from the same underlying parameter.

Table ~\ref{table:Multinomial} presents the non-negative ranks discovered by the algorithm under multinomial assumption. 

\begin{table}[]
\centering
\caption{Non-negative rank discovered by proposed algorithm under different levels of Gaussian noise. (Is it really non-negative rank, how can we be sure?)}
\label{table:Gaussian}
\begin{tabular}{|c|c|c|}
\hline
\textbf{\begin{tabular}[c]{@{}c@{}}$\beta$\\ (Noise level)\end{tabular}} & \textbf{\begin{tabular}[c]{@{}c@{}}$s$ \\ (Number of groups)\end{tabular}} & \textbf{Discovered rank} \\ \hline
0                                                                        & 200                                                                        & 20                       \\ \hline
0.1                                                                      & 150                                                                        & 20                       \\ \hline
0.2                                                                      & 125                                                                        & 19                       \\ \hline
0.3                                                                      & 95                                                                         & 21                       \\ \hline
0.4                                                                      & 93                                                                         & 22                       \\ \hline
0.5                                                                      & 120                                                                        & 21                       \\ \hline
\end{tabular}
\end{table}


\begin{table}[]
\centering
\caption{Identifying non-negative rank under Multinomial noise. Smaller $m$ implies higher noise.}
\label{table:Multinomial}
\begin{tabular}{|c|c|c|}
\hline
\textbf{\begin{tabular}[c]{@{}c@{}}$m$ \\ (sample size)\end{tabular}} & \textbf{\begin{tabular}[c]{@{}c@{}}$s$ \\ (Number of groups)\end{tabular}} & \textbf{Discovered rank} \\ \hline
1000                                                                  & 140                                                                        & 21                       \\ \hline
500                                                                   & 340                                                                        & 19                       \\ \hline
200                                                                   & 1000                                                                       & 33                       \\ \hline
\end{tabular}
\end{table}

\subsection{Experiments on real data}















%\section{Conclusion}

%
%\section*{References}
\bibliographystyle{icml2016}
\bibliography{nmf-bibliography}

\end{document}
